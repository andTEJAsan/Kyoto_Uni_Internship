% Gemini theme
% https://github.com/anishathalye/gemini
% We try to keep this Overleaf template in sync with the canonical source on
% GitHub, but it's recommended that you obtain the template directly from
% GitHub to ensure that you are using the latest version.

% Brown University color schemes for Gemini theme
% https://github.com/vskbellala/gemini-brown

\documentclass[final]{beamer}

% ====================
% Packages
% ====================

\usepackage[T1]{fontenc}
\usepackage{lmodern}
% \usepackage[size=custom,width=118.9,height=84.1,scale=1.3]{beamerposter}
\usepackage[size=a0,orientation=landscape,scale=1.3]{beamerposter}
\usetheme{gemini}
% \usecolortheme{albatross}
\usecolortheme{brownminimal}
\usepackage{graphicx}
\usepackage{booktabs}
\usepackage{tikz}
\usepackage{pgfplots}
\pgfplotsset{compat=1.14}
\usepackage{xcolor}
\usepackage{geometry}
\usepackage{setspace}
\usepackage{titlesec}
\usepackage{algorithmicx}
\usepackage{algorithm}
\usepackage{algpseudocode}
\usepackage{amsmath}
\usepackage{amsthm}

% ====================
% Lengths
% ====================

% If you have N columns, choose \sepwidth and \colwidth such that
% (N+1)*\sepwidth + N*\colwidth = \paperwidth
\newlength{\sepwidth}
\newlength{\colwidth}
\setlength{\sepwidth}{0.025\paperwidth}
\setlength{\colwidth}{0.3\paperwidth}

\newcommand{\separatorcolumn}{\begin{column}{\sepwidth}\end{column}}

% ====================
% Title
% ====================

\title{Approximate Model Counting and its applications in Quantitative Information Flow}

\author{Tejas Anand\inst{1} \and Kohei Suenaga\inst{2}}

\institute[shortinst]{IIT Delhi \inst{1} \samelineand Kyoto University \inst{2} }

% ====================
% Footer (optional)
% ====================

% \footercontent{
  % \href{https://www.example.com}{https://www.example.com} \hfill
  % ABC Conference 2025, New York --- XYZ-1234 \hfill
  % \href{mailto:alyssa.p.hacker@example.com}{alyssa.p.hacker@example.com}}
% (can be left out to remove footer)

% ====================
% Logo (optional)
% ====================

% use this to include logos on the left and/or right side of the header:
\logoright{\includegraphics[scale=0.6]{images/Kyoto_University_emblem.svg.png}}
\logoleft{\includegraphics[scale=0.6]{images/Indian_Institute_of_Technology_Delhi_Logo.svg.png}}
% \logoleft{\includegraphics[height=7cm]{images/Brown_full.png}}


% ====================
% Body
% ====================

\begin{document}

\begin{frame}[t]
\begin{columns}[t]
\separatorcolumn

\begin{column}{\colwidth}

  \begin{block}{What is Model Counting ? }

    % Some block contents, followed by a diagram, followed by a dummy paragraph.

    % \begin{figure}
    %   \centering
    %   \begin{tikzpicture}[scale=6]
    %     \draw[step=0.25cm,color=gray] (-1,-1) grid (1,1);
    %     \draw (1,0) -- (0.2,0.2) -- (0,1) -- (-0.2,0.2) -- (-1,0)
    %       -- (-0.2,-0.2) -- (0,-1) -- (0.2,-0.2) -- cycle;
    %   \end{tikzpicture}
    %   \caption{A figure caption.}
    % \end{figure}

    % Lorem ipsum dolor sit amet, consectetur adipiscing elit. Morbi ultricies
    % eget libero ac ullamcorper. Integer et euismod ante. Aenean vestibulum
    % lobortis augue, ut lobortis turpis rhoncus sed. Proin feugiat nibh a
    % lacinia dignissim. Proin scelerisque, risus eget tempor fermentum, ex
    % turpis condimentum urna, quis malesuada sapien arcu eu purus.
    	\begin{itemize}
		\item Model counting is the problem of counting the number of solutions to a given set of constraints.
		\item The problem of Model Counting (\#$\mathsf{SAT}$) is \#$\mathsf{P}$-complete. In very simple words, finding the \textcolor{red}{Exact Count} would take an abnormally high amount of time.
		\item Therefore, we work with an $(\epsilon,\delta)$ approximation algorithm $\mathcal{A}$, whose output $\mathsf{n}$ over a problem instance $\mathcal{F}$ satisfies,

	\end{itemize}
	\[
		\Pr[\mathsf{n} \leftarrow \mathcal{A}(\mathcal{F}) :  \frac{\#\mathcal{F}}{1 + \epsilon} \leq \mathsf{n} \leq \#\mathcal{F}(1 + \epsilon)  ]	 \geq  1 - \delta
	\]
	\begin{itemize}
		\item  In simple words, it gives a good enough number with high probability, for small values of $\epsilon$ and $\delta$.
		\item For instance, we might want to count the number of equivalence classes of the given relation
		\begin{equation}
			x \sim y \Leftrightarrow x \equiv y \equiv 0 \mod 8 \vee x = y
		\end{equation}
	\end{itemize}
  \end{block}

  \begin{block}{Research Problem}
    	\begin{itemize}
     
		\item Recently, a scalable approximation algorithm for model counting over boolean constraints was propsed by Chakraborty et al.
		\item We want to generalize this algorithm to simple arithmetic constraints like modulo, addition, subtraction, etc. over integers (finite fields like $Z_n$) and lists, using SMT solvers (SAT modulo theory) like Z3.
		\item This has applications in computer security, it would be the main ingredient to quantify the sensitive information leaked by a computer programme.
        \item In the internship so I proved the correctness for the algorithm over finite fields of integers $Z_k=\{0,1,\cdots k - 1\}$
	\end{itemize}

    % Nam vulputate nunc felis, non condimentum lacus porta ultrices. Nullam sed
    % sagittis metus. Etiam consectetur gravida urna quis suscipit.

    % \begin{itemize}
    %   \item \textbf{Mauris tempor} risus nulla, sed ornare
    %   \item \textbf{Libero tincidunt} a duis congue vitae
    %   \item \textbf{Dui ac pretium} morbi justo neque, ullamcorper
    % \end{itemize}

    % Eget augue porta, bibendum venenatis tortor.

  \end{block}

  \begin{block}{Chernoff Bounds}
  \begin{itemize}
      \item Chernoff Bounds are used to bound the probability of the value of a random variable lying outside a given window around the mean. A Chernoff Bound is used to prove the correctness of our Algorithm.
  \end{itemize}
	\textbf{Theorem. }Let $\Gamma$ be the sum of of r-wise independent random variables, each of
	which is confined to the interval $[0,1]$, and suppose that $E[\Gamma] = \mu$. For $0 < \beta 
	\leq 1$, if $r \leq \lfloor \beta^2 \mu e^{-1/2} \rfloor \leq 4$, then $\Pr[|\Gamma - \mu| > \beta \mu] \leq e^{-r/2} $

  \end{block}

\end{column}

\separatorcolumn

\begin{column}{\colwidth}

  \begin{block}{Example}
    	 \textbf{Question:} How many equivalence classes does the following relation have, where $x,y$ are 32 bit integers ? 
	\begin{equation}
		x \sim y \Leftrightarrow x \equiv y \equiv 0 \mod 8 \textbf{ or } x = y
	\end{equation}
	% \pause
	\begin{center}
		\textbf{Answer:} $7\cdot2^{29} + 1$. All of the multiples of 8 form 1 equivalence class, and the remaining $7/8^{ths}$ of the total $2^{32}$ integers form singleton equivalence classes of their own.
	\end{center}
	% \pause
	\begin{center}
		$(x \equiv y \equiv 0 \mod 8) \vee (x = y) \rightarrow$ \fbox{\textbf{Our Algorithm}}  $\rightsquigarrow 7\cdot2^{29}$
	\end{center}
    % Vivamus congue volutpat elit non semper. Praesent molestie nec erat ac
    % interdum. In quis suscipit erat. \textbf{Phasellus mauris felis, molestie
    % ac pharetra quis}, tempus nec ante. Donec finibus ante vel purus mollis
    % fermentum. Sed felis mi, pharetra eget nibh a, feugiat eleifend dolor. Nam
    % mollis condimentum purus quis sodales. Nullam eu felis eu nulla eleifend
    % bibendum nec eu lorem. Vivamus felis velit, volutpat ut facilisis ac,
    % commodo in metus.

    % \begin{enumerate}
    %   \item \textbf{Morbi mauris purus}, egestas at vehicula et, convallis
    %     accumsan orci. Orci varius natoque penatibus et magnis dis parturient
    %     montes, nascetur ridiculus mus.
    %   \item \textbf{Cras vehicula blandit urna ut maximus}. Aliquam blandit nec
    %     massa ac sollicitudin. Curabitur cursus, metus nec imperdiet bibendum,
    %     velit lectus faucibus dolor, quis gravida metus mauris gravida turpis.
    %   \item \textbf{Vestibulum et massa diam}. Phasellus fermentum augue non
    %     nulla accumsan, non rhoncus lectus condimentum.
    % \end{enumerate}

  \end{block}

  \begin{block}{Polynomially Many Queries to the oracle SMT}

    \begin{itemize}
        \item For simplicity, we have an oracle $\mathsf{SMT}$, which on invocation over an arithmetic formula $\mathcal{F}$, gives us any one solution to it or tells if it is $\mathsf{UNSAT}$.
        \item The model count, $#\mathcal{F}$ can be exponentially large, so the naive approach of invoking the oracle till we get no new solutions is infeasible.
        \item A better way is to keep adding the negation of all the models we have obtained so far to the formula $\mathcal{F}$ as $\mathcal{F} \leftarrow \mathcal{F} \land (x \neq x_0) \land (x \neq x_1) \cdots (x \neq x_i)$, where $x_0, x_1, \cdots x_i$ are the models obtained so far. This approach is better than the first one, but it is still exponential.
    \end{itemize}
  \end{block}

  \begin{block}{How ApproxMC works, intuitively}
  
    \begin{itemize}
        \item We must take advantage of the fact that we are not concerened with the exact model count.
        \item So we divide the set of models uniformly into \textbf{cells}, using a randomly sampled $r-$wise independent hash function. 
        \item All models having a given hash-value belong to a given cell.
        \item The representative count of a given cell is then multiplied by the number of cells to get an approximate count of the total number of models of the formula.
        \item We try to make the number of cells such that it is "fine grained" enough and has $\leq pivot$ models in it.
        \item This $pivot$ is a function of the threshold $\varepsilon$ around the correct model that we desire. It is of the order $\mathcal{O}(1/\varepsilon^2)$.
        \item Hence, we only need to make a bounded number of invocations to the oracle $\mathsf{SMT}$.
    \end{itemize}
    
  \end{block}

\end{column}

\separatorcolumn

\begin{column}{\colwidth}

  \begin{block}{}
\begin{algorithm}[H]
	\caption{$\mathsf{ApproxMCCore}(F,pivot)$}
\begin{algorithmic}[1]
	\State $S \gets \mathsf{BoundedSMT}(F, pivot + 1)$
		\Comment{Assume $x_1, x_2 \cdots x_q$ are the variables of $F$}
	\If{$|S| \leq pivot$}
		\State \textbf{return} $|S|$;
	\Else
		% \State $m \gets m + 1$
		% \State Choose a new hash function from $\mathcal{H}_k(n, m, r)$
		\State $l \gets \lfloor log_{k}(pivot) \rfloor - 1$; $i \gets l - 1$
		\Repeat
		\State $i \gets i + 1$;
		\State Choose $h \gets \mathcal{H}_k(n, i - l , 3)$ uniformly at random;
		\State Choose $\alpha \gets Z_k^{i - l}$ uniformly at random;
		\State $S \gets \mathsf{BoundedSMT}(F \wedge h(x_1,x_2\cdots x_q) = \alpha, pivot + 1)$
		\Until{$(1 \leq |S| \leq pivot)$ or $(i = n)$};
	\EndIf
	% \If{} 
	% % 	% \State \textbf{return} $\bot$;
	% \Else 
	% \EndIf
	\If{$(|S| > pivot$ \textbf{ or } $|S| = 0)$}
		\State \textbf{return} $\bot$;
	\Else 
		\State \textbf{return} $|S|\cdot k ^{i - l}$;
	\EndIf

    \end{algorithmic}
\end{algorithm}

    $\mathcal{H}_k(n,m,r)$ = Set of hash functions from $Z_k^{n} \rightarrow Z_k^{m}$ which are  $r$-wise independent
    The no. of cells is equal to the number of unique hash values possible which is equal to $|Z_k^m| = k^m$
  \end{block}

  \begin{block}{Quantitative Information Flow}

    \begin{itemize}
        \item We often require the number of equivalence classes to calculate the \textbf{Bayes Vulnerability} which is an Adversary's maximum probability of guessing the correct "secret" given he knows the encryption scheme.
        \item For some information theoretic measures, we even require the size of each equivalence class.
        \item Our work can be used to quantify the sensitive information leaked by a computer program.
    \end{itemize}

  \end{block}

  \begin{block}{References}

    \nocite{*}
    \footnotesize{\bibliographystyle{abbrv}\bibliography{poster}}

  \end{block}

\end{column}

\separatorcolumn
\end{columns}
\end{frame}

\end{document}